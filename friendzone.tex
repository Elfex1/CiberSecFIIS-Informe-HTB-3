\subsection{HTB04 - Friendzone}

    \subsubsection{Escaneo}
        \large{Como primera etapa de la Prueba de Penetración realizamos un escaneo de puertos abiertos en la máquina víctima con la herramienta "Nmap", donde se encuentran siete de ellos.}
        \par
        \begin{figure}[H]
            \centering
            \includegraphics[width=0.99\textwidth]{informe4/imagenes/friendzone/01_escaneo.png}
            \caption{Escaneo de puertos Friendzone} 
        \end{figure}  

    \subsubsection{Análisis de Vulnerabilidades}
        \large{Se intenta acceder a la máquina mediante el puerto 'ftp', pero no se consigue el acceso.}
        \par
        \begin{figure}[H]
            \centering
            \includegraphics[width=0.99\textwidth]{informe4/imagenes/friendzone/02_ftp.png}
            \caption{Puerto ftp en Friendzone} 
        \end{figure}
        
        \large{Luego probamos con el puerto 'netbios ssn', usando la herramienta 'smbmap'.}
        \par
        \begin{figure}[H]
            \centering
            \includegraphics[width=0.99\textwidth]{informe4/imagenes/friendzone/03_smbmap.png}
            \caption{Hallazgo de la herramienta smbmap en Friendzone} 
        \end{figure}

        \large{Identificamos dos carpeta a las que se tiene acceso, una llamada genarl con acceso solo para leer y otra con nombre Development con accesos para leer y escribir. Revisamos la carpeta general.}
        \par
        \begin{figure}[H]
            \centering
            \includegraphics[width=0.99\textwidth]{informe4/imagenes/friendzone/04_smb_creds.png}
            \caption{Carpeta general en Friendzone} 
        \end{figure}

        \large{Dentro de la carpeta general encontramos una archivo '.txt' con el nombre creds, el cual parece una abreviacion de credenciales, por lo cual revisamos el archivo y al hacerlo encontramos las credenciales de admin.}
        \par
        \begin{figure}[H]
            \centering
            \includegraphics[width=0.99\textwidth]{informe4/imagenes/friendzone/05_cred_friendzone.png}
            \caption{Archivo creds.txt en Friendzone} 
        \end{figure}

        \large{Se revisó el servicio web, donde se encontró información de dominio para correo electrónico.}
        \begin{figure}[H]
            \centering
            \includegraphics[width=0.8\textwidth]{informe4/imagenes/friendzone/06_web.png}
            \caption{Página web de la máquina Friendzone} 
        \end{figure}

        \large{Se procedió a agregar el dominio encontrado en la página web al archivo /etc/hosts con la finalidad de obtener acceso a los servicios almacenados.}
        \begin{figure}[H]
            \centering
            \includegraphics[width=0.99\textwidth]{informe4/imagenes/friendzone/07_etchosts.png}
            \caption{Dominio friendzone agregado al archivo /etc/hosts} 
        \end{figure}

        \begin{figure}[H]
            \centering
            \includegraphics[width=0.99\textwidth]{informe4/imagenes/friendzone/08_ffuf.png}
            \caption{Enumeración de directorios en la página web} 
        \end{figure}

        \begin{figure}[H]
            \centering
            \includegraphics[width=0.99\textwidth]{informe4/imagenes/friendzone/09_https_friendzoneportal.png}
            \caption{Página https friendzoneportal} 
        \end{figure}

        \begin{figure}[H]
            \centering
            \includegraphics[width=0.99\textwidth]{informe4/imagenes/friendzone/10_https_friendzone.png}
            \caption{Página https friendzone} 
        \end{figure}

        \begin{figure}[H]
            \centering
            \includegraphics[width=0.99\textwidth]{informe4/imagenes/friendzone/11_sublister1.png}
            \caption{Enumeración de subdominios en friendzoneportal} 
        \end{figure}

        \begin{figure}[H]
            \centering
            \includegraphics[width=0.99\textwidth]{informe4/imagenes/friendzone/12_sublister2.png}
            \caption{Enumeración de subdominios en friendzone} 
        \end{figure}

        \begin{figure}[H]
            \centering
            \includegraphics[width=0.99\textwidth]{informe4/imagenes/friendzone/13_administratorpage.png}
            \caption{Login form en Friendzone} 
        \end{figure}

        \begin{figure}[H]
            \centering
            \includegraphics[width=0.99\textwidth]{informe4/imagenes/friendzone/14_login.png}
            \caption{Acceso login como Administrador} 
        \end{figure}

        \begin{figure}[H]
            \centering
            \includegraphics[width=0.99\textwidth]{informe4/imagenes/friendzone/15_dashboard.png}
            \caption{Ruta dashboard en la página Friendzone} 
        \end{figure}

        \begin{figure}[H]
            \centering
            \includegraphics[width=0.99\textwidth]{informe4/imagenes/friendzone/16_dashboardimage.png}
            \caption{Dominio friendzone agregado al archivo /etc/hosts} 
        \end{figure}

    \subsubsection{Explotación}

    \begin{figure}[H]
        \centering
        \includegraphics[width=0.99\textwidth]{informe4/imagenes/friendzone/17_reverse_shell_upload.png}
        \caption{Dominio friendzone agregado al archivo /etc/hosts} 
    \end{figure}

    \begin{figure}[H]
        \centering
        \includegraphics[width=0.99\textwidth]{informe4/imagenes/friendzone/18_consult.png}
        \caption{Dominio friendzone agregado al archivo /etc/hosts} 
    \end{figure}

    \begin{figure}[H]
        \centering
        \includegraphics[width=0.99\textwidth]{informe4/imagenes/friendzone/19_conexion_wwwdata.png}
        \caption{Dominio friendzone agregado al archivo /etc/hosts} 
    \end{figure}
    \subsubsection{Escalamiento de Privilegios}

    \subsubsection{Post-Explotación}

    \subsubsection{Recomendaciones de Mitigación}
